\documentclass[danmark]{../AU}
\usepackage[danish]{babel}
\usepackage[style=danish]{csquotes}
\MakeOuterQuote{"}

\title{Uge 1}
\author{Jacob Emil Ulvedal Rosborg}
\date{august 2018}

\begin{document}
\section*{Opgave 4}
Mulige udfald for to seks siddet terninger

\begin{equation}
  udfald = 
  \begin{Bmatrix}
  (1,1) & (1,2) & \cdots & (1,6) \\
  (2,1) & (2,2) & \cdots & (2,6) \\
  \vdots  & \vdots  & \ddots & \vdots  \\
  (6,1) & (6,2) & \cdots & (6,6) 
 \end{Bmatrix}
\end{equation}

Grundet symmetrien af vores udfaldsrum finder vi at sandsyneligheden $P(n)$ for summen $n$ af to terninger er

\begin{equation}
  P(n) = 
  \begin{cases} 
      \frac{n - 1}{36} & 0 < n \leq 7 \\
      \frac{13 - n}{36} & 7 < x \leq 12
   \end{cases}
   \quad n \in \mathbb{N} : 0 < n \leq 12
\end{equation}

Herved finder vi at gennemsnittet for udfaldsrummet er

\begin{align}
  average &= \sum n P(n) \Rightarrow \\
  average &= 1 P(1) + 2 P(2) + \ldots + 12 P(12) = 7
\end{align}

\end{document}
